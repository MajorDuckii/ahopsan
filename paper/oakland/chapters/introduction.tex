\section{Introduction}

Address sanitizer is very useful but introduces a high overhead around ~2x.
Thus can only be applied in the program testing stage, as a companion to the fuzzing campaign.

We hope it can be applied in production.
It is more generic than other security tools like control-flow integrity, stack canary, and shadow stacks.


The key idea is to avoid unnecessary checking. Especially for loops.
While not skip necessary checks.

1. First find a test program with loops.
2. Modify at the source code level to enforce shops, and evaluate its overhead.


\begin{table}[tp]
    \begin{tabular}{|l|l|l|l|l|}
        \hline
                     & Simple Iterating & Factorial Calculation & Euclidean GCD & Fibonacci sequence \\ \hline
        Simple Loop  &                  &                       &               &                    \\ \hline
        Double Loop  &                  &                       &               &                    \\ \hline
        Tripper Loop &                  &                       &               &                    \\ \hline
    \end{tabular}
    \caption{Performance evaluation on micro-benchmarks.}
\end{table}

In order to ;
shadow the variables,

We need to know the poiters.
Just like
we need to
If there are
then how to shadow the pointers.


It works like to adjust the source code. to enrich the source code.
1. How to identify data pointers;
2. How to instrument; base on LLVM; This should be accomplished by front-end.
3. How to do check;
4. The affection of compiler optimization.




