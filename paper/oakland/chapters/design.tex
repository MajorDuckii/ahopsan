\section{System Design}


\newcounter{dsgn}
\renewcommand\yphn[1]{\stepcounter{dsgn} \textcolor{blue} {\thedsgn.} \yphl{#1}}


We design a semantic-aware memory access profiler and
a semantic-aware address sanitizer.

\subsection{Languate Level Semantics}
To parse the semantics of each memory access is critical.

Currently, we define the following semantics: array access or struct access, pointer dereference.

\yphn{Inputs}

We exploit the source code.

\yphn{Outputs}
For a program, both profiler and sanitizer would generate their binaries.


\subsection{Semantic Analysis}
An important task is to collect the semantics of memory access.

\subsection{Memory object organization}
Do we need to use variable size red-zones for different kinds of objects?

\subsubsection{Heap Buffers}

\subsubsection{Stack buffers}

\subsubsection{Internal buffers}
An object sometimes contains a buffer inside itself.
This buffer may be security-sensitive if its adjacent members are function pointers.

For such an object, we create red-zones to surround the buffers, similar to what Asan does on stack buffers.

\subsection{Instrumentation}

Do we need to do another check after a loop?

\subsection{Exception Handling}


\begin{minted}[fontsize=\scriptsize,frame=single,linenos,numbersep=-6pt]{llvm}
  entry:
    %p.addr = alloca i8*, align 8
    store i8* %p, i8** %p.addr, align 8
    br label %while.cond  
  while.cond:
    %0 = load i8*, i8** %p.addr, align 8
    %1 = load i8, i8* %0, align 1
    %tobool = icmp ne i8 %1, 0
    br i1 %tobool, label %while.body, label %while.end
  while.body:    
    %2 = load i8*, i8** %p.addr, align 8
    %3 = load i8, i8* %2, align 1
    %conv = sext i8 %3 to i32
    %4 = load i32, i32* @globalx, align 4
    %add = add nsw i32 %4, %conv
    store i32 %add, i32* @globalx, align 4
    %5 = load i8*, i8** %p.addr, align 8
    %incdec.ptr = getelementptr inbounds i8, i8* %5, i32 1
    store i8* %incdec.ptr, i8** %p.addr, align 8
    br label %while.cond  
  while.end:
    ret void
\end{minted}

A sound algorithm.

Symbolic address-set Analsysis based on Symbolic IR execution.

This is really an old problem.



\subsection{Symbolic IR Execution}

% The design of symbolic execution engine
% Input & output

The address set of each IR register.
LLVM-IR has unlimited registesrs.

Algorithm:
The exeuction enginne schedule the execution basic block by block.
The schedule units are basiuc block.
Each basic block are feed by a seriours of configuartion.
Each configuration is consist with a set of rgistesrs status.
The entry basick block start with an empty basic block.
The partermeters.
Eveay new occurentcy of a regiuster is denoted by a symbol value.
Every IR instructoin is.
The register staus changes along with the IR execution.

Machine status: Register status. + Memory status.

Change the status along the IR execution.

All registesrs and memroy slots are initialized with symbolic values.
Therefore, the first read of a registe or a memoery adddress would get a symbolic values.

Every Function is an execution unit.




\subsection{Symbolic Address-set Analysis}
For IR instruction, simply idenfitied by the line number.
On my symbolic execution engine, collect the accessed address by each instruction.

Then analsys the memory access patterns, to
idenitfy array access an struct member access.




\subsection{Instrumentation}

\subsection{Data Layout}



