%% Modified for NDSS 2020 by DB on 2019/05/15
%%
%% bare_conf.tex
%% V1.3
%% 2007/01/11
%% by Michael Shell
%% See:
%% http://www.michaelshell.org/
%% for current contact information.
%%
%% This is a skeleton file demonstrating the use of IEEEtran.cls
%% (requires IEEEtran.cls version 1.7 or later) with an IEEE conference paper.
%%
%% Support sites:
%% http://www.michaelshell.org/tex/ieeetran/
%% http://www.ctan.org/tex-archive/macros/latex/contrib/IEEEtran/
%% and
%% http://www.ieee.org/

%%*************************************************************************
%% Legal Notice:
%% This code is offered as-is without any warranty either expressed or
%% implied; without even the implied warranty of MERCHANTABILITY or
%% FITNESS FOR A PARTICULAR PURPOSE! 
%% User assumes all risk.
%% In no event shall IEEE or any contributor to this code be liable for
%% any damages or losses, including, but not limited to, incidental,
%% consequential, or any other damages, resulting from the use or misuse
%% of any information contained here.
%%
%% All comments are the opinions of their respective authors and are not
%% necessarily endorsed by the IEEE.
%%
%% This work is distributed under the LaTeX Project Public License (LPPL)
%% ( http://www.latex-project.org/ ) version 1.3, and may be freely used,
%% distributed and modified. A copy of the LPPL, version 1.3, is included
%% in the base LaTeX documentation of all distributions of LaTeX released
%% 2003/12/01 or later.
%% Retain all contribution notices and credits.
%% ** Modified files should be clearly indicated as such, including  **
%% ** renaming them and changing author support contact information. **
%%
%% File list of work: IEEEtran.cls, IEEEtran_HOWTO.pdf, bare_adv.tex,
%%                    bare_conf.tex, bare_jrnl.tex, bare_jrnl_compsoc.tex
%%*************************************************************************

% *** Authors should verify (and, if needed, correct) their LaTeX system  ***
% *** with the testflow diagnostic prior to trusting their LaTeX platform ***
% *** with production work. IEEE's font choices can trigger bugs that do  ***
% *** not appear when using other class files.                            ***
% The testflow support page is at:
% http://www.michaelshell.org/tex/testflow/



% Note that the a4paper option is mainly intended so that authors in
% countries using A4 can easily print to A4 and see how their papers will
% look in print - the typesetting of the document will not typically be
% affected with changes in paper size (but the bottom and side margins will).
% Use the testflow package mentioned above to verify correct handling of
% both paper sizes by the user's LaTeX system.
%
% Also note that the "draftcls" or "draftclsnofoot", not "draft", option
% should be used if it is desired that the figures are to be displayed in
% draft mode.
%
\documentclass[conference]{IEEEtran}
% Add the compsoc option for Computer Society conferences.
%
% If IEEEtran.cls has not been installed into the LaTeX system files,
% manually specify the path to it like:
% \documentclass[conference]{../sty/IEEEtran}

\pagestyle{plain}


% Some very useful LaTeX packages include:
% (uncomment the ones you want to load)


% *** MISC UTILITY PACKAGES ***
%
%\usepackage{ifpdf}
% Heiko Oberdiek's ifpdf.sty is very useful if you need conditional
% compilation based on whether the output is pdf or dvi.
% usage:
% \ifpdf
%   % pdf code
% \else
%   % dvi code
% \fi
% The latest version of ifpdf.sty can be obtained from:
% http://www.ctan.org/tex-archive/macros/latex/contrib/oberdiek/
% Also, note that IEEEtran.cls V1.7 and later provides a builtin
% \ifCLASSINFOpdf conditional that works the same way.
% When switching from latex to pdflatex and vice-versa, the compiler may
% have to be run twice to clear warning/error messages.






% *** CITATION PACKAGES ***
%
%\usepackage{cite}
% cite.sty was written by Donald Arseneau
% V1.6 and later of IEEEtran pre-defines the format of the cite.sty package
% \cite{} output to follow that of IEEE. Loading the cite package will
% result in citation numbers being automatically sorted and properly
% "compressed/ranged". e.g., [1], [9], [2], [7], [5], [6] without using
% cite.sty will become [1], [2], [5]--[7], [9] using cite.sty. cite.sty's
% \cite will automatically add leading space, if needed. Use cite.sty's
% noadjust option (cite.sty V3.8 and later) if you want to turn this off.
% cite.sty is already installed on most LaTeX systems. Be sure and use
% version 4.0 (2003-05-27) and later if using hyperref.sty. cite.sty does
% not currently provide for hyperlinked citations.
% The latest version can be obtained at:
% http://www.ctan.org/tex-archive/macros/latex/contrib/cite/
% The documentation is contained in the cite.sty file itself.






% *** GRAPHICS RELATED PACKAGES ***
%
\ifCLASSINFOpdf
  % \usepackage[pdftex]{graphicx}
  % declare the path(s) where your graphic files are
  % \graphicspath{{../pdf/}{../jpeg/}}
  % and their extensions so you won't have to specify these with
  % every instance of \includegraphics
  % \DeclareGraphicsExtensions{.pdf,.jpeg,.png}
\else
  % or other class option (dvipsone, dvipdf, if not using dvips). graphicx
  % will default to the driver specified in the system graphics.cfg if no
  % driver is specified.
  % \usepackage[dvips]{graphicx}
  % declare the path(s) where your graphic files are
  % \graphicspath{{../eps/}}
  % and their extensions so you won't have to specify these with
  % every instance of \includegraphics
  % \DeclareGraphicsExtensions{.eps}
\fi
% graphicx was written by David Carlisle and Sebastian Rahtz. It is
% required if you want graphics, photos, etc. graphicx.sty is already
% installed on most LaTeX systems. The latest version and documentation can
% be obtained at: 
% http://www.ctan.org/tex-archive/macros/latex/required/graphics/
% Another good source of documentation is "Using Imported Graphics in
% LaTeX2e" by Keith Reckdahl which can be found as epslatex.ps or
% epslatex.pdf at: http://www.ctan.org/tex-archive/info/
%
% latex, and pdflatex in dvi mode, support graphics in encapsulated
% postscript (.eps) format. pdflatex in pdf mode supports graphics
% in .pdf, .jpeg, .png and .mps (metapost) formats. Users should ensure
% that all non-photo figures use a vector format (.eps, .pdf, .mps) and
% not a bitmapped formats (.jpeg, .png). IEEE frowns on bitmapped formats
% which can result in "jaggedy"/blurry rendering of lines and letters as
% well as large increases in file sizes.
%
% You can find documentation about the pdfTeX application at:
% http://www.tug.org/applications/pdftex





% *** MATH PACKAGES ***
%
%\usepackage[cmex10]{amsmath}
% A popular package from the American Mathematical Society that provides
% many useful and powerful commands for dealing with mathematics. If using
% it, be sure to load this package with the cmex10 option to ensure that
% only type 1 fonts will utilized at all point sizes. Without this option,
% it is possible that some math symbols, particularly those within
% footnotes, will be rendered in bitmap form which will result in a
% document that can not be IEEE Xplore compliant!
%
% Also, note that the amsmath package sets \interdisplaylinepenalty to 10000
% thus preventing page breaks from occurring within multiline equations. Use:
%\interdisplaylinepenalty=2500
% after loading amsmath to restore such page breaks as IEEEtran.cls normally
% does. amsmath.sty is already installed on most LaTeX systems. The latest
% version and documentation can be obtained at:
% http://www.ctan.org/tex-archive/macros/latex/required/amslatex/math/





% *** SPECIALIZED LIST PACKAGES ***
%
%\usepackage{algorithmic}
% algorithmic.sty was written by Peter Williams and Rogerio Brito.
% This package provides an algorithmic environment fo describing algorithms.
% You can use the algorithmic environment in-text or within a figure
% environment to provide for a floating algorithm. Do NOT use the algorithm
% floating environment provided by algorithm.sty (by the same authors) or
% algorithm2e.sty (by Christophe Fiorio) as IEEE does not use dedicated
% algorithm float types and packages that provide these will not provide
% correct IEEE style captions. The latest version and documentation of
% algorithmic.sty can be obtained at:
% http://www.ctan.org/tex-archive/macros/latex/contrib/algorithms/
% There is also a support site at:
% http://algorithms.berlios.de/index.html
% Also of interest may be the (relatively newer and more customizable)
% algorithmicx.sty package by Szasz Janos:
% http://www.ctan.org/tex-archive/macros/latex/contrib/algorithmicx/




% *** ALIGNMENT PACKAGES ***
%
%\usepackage{array}
% Frank Mittelbach's and David Carlisle's array.sty patches and improves
% the standard LaTeX2e array and tabular environments to provide better
% appearance and additional user controls. As the default LaTeX2e table
% generation code is lacking to the point of almost being broken with
% respect to the quality of the end results, all users are strongly
% advised to use an enhanced (at the very least that provided by array.sty)
% set of table tools. array.sty is already installed on most systems. The
% latest version and documentation can be obtained at:
% http://www.ctan.org/tex-archive/macros/latex/required/tools/


%\usepackage{mdwmath}
%\usepackage{mdwtab}
% Also highly recommended is Mark Wooding's extremely powerful MDW tools,
% especially mdwmath.sty and mdwtab.sty which are used to format equations
% and tables, respectively. The MDWtools set is already installed on most
% LaTeX systems. The lastest version and documentation is available at:
% http://www.ctan.org/tex-archive/macros/latex/contrib/mdwtools/


% IEEEtran contains the IEEEeqnarray family of commands that can be used to
% generate multiline equations as well as matrices, tables, etc., of high
% quality.


%\usepackage{eqparbox}
% Also of notable interest is Scott Pakin's eqparbox package for creating
% (automatically sized) equal width boxes - aka "natural width parboxes".
% Available at:
% http://www.ctan.org/tex-archive/macros/latex/contrib/eqparbox/





% *** SUBFIGURE PACKAGES ***
%\usepackage[tight,footnotesize]{subfigure}
% subfigure.sty was written by Steven Douglas Cochran. This package makes it
% easy to put subfigures in your figures. e.g., "Figure 1a and 1b". For IEEE
% work, it is a good idea to load it with the tight package option to reduce
% the amount of white space around the subfigures. subfigure.sty is already
% installed on most LaTeX systems. The latest version and documentation can
% be obtained at:
% http://www.ctan.org/tex-archive/obsolete/macros/latex/contrib/subfigure/
% subfigure.sty has been superceeded by subfig.sty.



%\usepackage[caption=false]{caption}
%\usepackage[font=footnotesize]{subfig}
% subfig.sty, also written by Steven Douglas Cochran, is the modern
% replacement for subfigure.sty. However, subfig.sty requires and
% automatically loads Axel Sommerfeldt's caption.sty which will override
% IEEEtran.cls handling of captions and this will result in nonIEEE style
% figure/table captions. To prevent this problem, be sure and preload
% caption.sty with its "caption=false" package option. This is will preserve
% IEEEtran.cls handing of captions. Version 1.3 (2005/06/28) and later 
% (recommended due to many improvements over 1.2) of subfig.sty supports
% the caption=false option directly:
%\usepackage[caption=false,font=footnotesize]{subfig}
%
% The latest version and documentation can be obtained at:
% http://www.ctan.org/tex-archive/macros/latex/contrib/subfig/
% The latest version and documentation of caption.sty can be obtained at:
% http://www.ctan.org/tex-archive/macros/latex/contrib/caption/




% *** FLOAT PACKAGES ***
%
%\usepackage{fixltx2e}
% fixltx2e, the successor to the earlier fix2col.sty, was written by
% Frank Mittelbach and David Carlisle. This package corrects a few problems
% in the LaTeX2e kernel, the most notable of which is that in current
% LaTeX2e releases, the ordering of single and double column floats is not
% guaranteed to be preserved. Thus, an unpatched LaTeX2e can allow a
% single column figure to be placed prior to an earlier double column
% figure. The latest version and documentation can be found at:
% http://www.ctan.org/tex-archive/macros/latex/base/



%\usepackage{stfloats}
% stfloats.sty was written by Sigitas Tolusis. This package gives LaTeX2e
% the ability to do double column floats at the bottom of the page as well
% as the top. (e.g., "\begin{figure*}[!b]" is not normally possible in
% LaTeX2e). It also provides a command:
%\fnbelowfloat
% to enable the placement of footnotes below bottom floats (the standard
% LaTeX2e kernel puts them above bottom floats). This is an invasive package
% which rewrites many portions of the LaTeX2e float routines. It may not work
% with other packages that modify the LaTeX2e float routines. The latest
% version and documentation can be obtained at:
% http://www.ctan.org/tex-archive/macros/latex/contrib/sttools/
% Documentation is contained in the stfloats.sty comments as well as in the
% presfull.pdf file. Do not use the stfloats baselinefloat ability as IEEE
% does not allow \baselineskip to stretch. Authors submitting work to the
% IEEE should note that IEEE rarely uses double column equations and
% that authors should try to avoid such use. Do not be tempted to use the
% cuted.sty or midfloat.sty packages (also by Sigitas Tolusis) as IEEE does
% not format its papers in such ways.





% *** PDF, URL AND HYPERLINK PACKAGES ***
%
%\usepackage{url}
% url.sty was written by Donald Arseneau. It provides better support for
% handling and breaking URLs. url.sty is already installed on most LaTeX
% systems. The latest version can be obtained at:
% http://www.ctan.org/tex-archive/macros/latex/contrib/misc/
% Read the url.sty source comments for usage information. Basically,
% \url{my_url_here}.


\usepackage{color}
\newcommand{\yphl}[1]{\textcolor{blue} {#1}\\}
\newcommand{\yphn}[1]{\yphl{#1}}
\usepackage{minted}

% \usepackage[margin=1.5in]{geometry}    % For reducing margin
% \usepackage[english]{babel}
% \usepackage[utf8]{inputenc}
\usepackage{algorithm}
\usepackage{algorithmicx}
%\usepackage[algo2e]{algorithm2e} 
% \usepackage[linesnumbered,ruled,vlined]{algorithm2e}
% \usepackage{arevmath}     % For math symbols
% \usepackage[noend]{algpseudocode}
\usepackage{algpseudocode}
% \usepackage{algorithmic,algorithm2e,float}
% \algsetup{linenosize=\tiny}
% \usepackage{float}
\algdef{SE}[DOWHILE]{Do}{doWhile}{\algorithmicdo}[1]{\algorithmicwhile\ #1}%
% New definitions
\algnewcommand\algorithmicswitch{\textbf{switch}}
\algnewcommand\algorithmiccase{\textbf{case}}
\algnewcommand\algorithmicassert{\texttt{assert}}
\algnewcommand\Assert[1]{\State \algorithmicassert(#1)}%
% New "environments"
\algdef{SE}[SWITCH]{Switch}{EndSwitch}[1]{\algorithmicswitch\ #1\ \algorithmicdo}{\algorithmicend\ \algorithmicswitch}%
\algdef{SE}[CASE]{Case}{EndCase}[1]{\algorithmiccase\ #1}{\algorithmicend\ \algorithmiccase}%
\algtext*{EndSwitch}%
\algtext*{EndCase}%

% *** Do not adjust lengths that control margins, column widths, etc. ***
% *** Do not use packages that alter fonts (such as pslatex).         ***
% There should be no need to do such things with IEEEtran.cls V1.6 and later.
% (Unless specifically asked to do so by the journal or conference you plan
% to submit to, of course. )


% correct bad hyphenation here
\hyphenation{op-tical net-works semi-conduc-tor}


\begin{document}
%
% paper title
% can use linebreaks \\ within to get better formatting as desired
\title{AHopSan: A Fast Semantic-aware Address Sanitizer \\
  \huge {-- Be Pragmatist Not Perfectionist}}

% author names and affiliations
% use a multiple column layout for up to three different
% affiliations
% \author{\IEEEauthorblockN{Michael Shell}
% \IEEEauthorblockA{Georgia Institute of Technology\\
% someemail@somedomain.com}
% \and
% \IEEEauthorblockN{Homer Simpson}
% \IEEEauthorblockA{Twentieth Century Fox\\
% homer@thesimpsons.com}
% \and
% \IEEEauthorblockN{James Kirk\\ and Montgomery Scott}
% \IEEEauthorblockA{Starfleet Academy\\
% someemail@somedomain.com}}

% conference papers do not typically use \thanks and this command
% is locked out in conference mode. If really needed, such as for
% the acknowledgment of grants, issue a \IEEEoverridecommandlockouts
% after \documentclass

% for over three affiliations, or if they all won't fit within the width
% of the page, use this alternative format:
% 
%\author{\IEEEauthorblockN{Michael Shell\IEEEauthorrefmark{1},
%Homer Simpson\IEEEauthorrefmark{2},
%James Kirk\IEEEauthorrefmark{3}, 
%Montgomery Scott\IEEEauthorrefmark{3} and
%Eldon Tyrell\IEEEauthorrefmark{4}}
%\IEEEauthorblockA{\IEEEauthorrefmark{1}School of Electrical and Computer Engineering\\
%Georgia Institute of Technology,
%Atlanta, Georgia 30332--0250\\ Email: see http://www.michaelshell.org/contact.html}
%\IEEEauthorblockA{\IEEEauthorrefmark{2}Twentieth Century Fox, Springfield, USA\\
%Email: homer@thesimpsons.com}
%\IEEEauthorblockA{\IEEEauthorrefmark{3}Starfleet Academy, San Francisco, California 96678-2391\\
%Telephone: (800) 555--1212, Fax: (888) 555--1212}
%\IEEEauthorblockA{\IEEEauthorrefmark{4}Tyrell Inc., 123 Replicant Street, Los Angeles, California 90210--4321}}




% use for special paper notices
%\IEEEspecialpapernotice{(Invited Paper)}



\IEEEoverridecommandlockouts
\makeatletter\def\@IEEEpubidpullup{6.5\baselineskip}\makeatother
\IEEEpubid{\parbox{\columnwidth}{
    Network and Distributed Systems Security (NDSS) Symposium 2020\\
    23-26 February 2020, San Diego, CA, USA\\
    ISBN 1-891562-61-4\\
    https://dx.doi.org/10.14722/ndss.2020.23xxx\\
    www.ndss-symposium.org
  }
  \hspace{\columnsep}\makebox[\columnwidth]{}}


% make the title area
\maketitle


\begin{abstract}
    %\boldmath
    The goal of this work.
\end{abstract}
% IEEEtran.cls defaults to using nonbold math in the Abstract.
% This preserves the distinction between vectors and scalars. However,
% if the conference you are submitting to favors bold math in the abstract,
% then you can use LaTeX's standard command \boldmath at the very start
% of the abstract to achieve this. Many IEEE journals/conferences frown on
% math in the abstract anyway.

% no keywords




% For peer review papers, you can put extra information on the cover
% page as needed:
% \ifCLASSOPTIONpeerreview
% \begin{center} \bfseries EDICS Category: 3-BBND \end{center}
% \fi
%
% For peerreview papers, this IEEEtran command inserts a page break and
% creates the second title. It will be ignored for other modes.
%%\IEEEpeerreviewmaketitle

\section{Introduction}

\newcounter{intr}
\renewcommand\yphn[1]{\stepcounter{intr} \textcolor{blue} {\theintr.} \yphl{#1}}

\yphn{Memory errors}

According to a recent report that examining 25 years of vulnerabilities (from 1998 to 2012),
buffer overflow causes 14 percent of software security vulnerabilities and 35 percent of critical vulnerabilities,
making it the leading cause of software security vulnerabilities overall.

As of July 2014, the TIOBE index indicates that the C programming language,
which is the language most commonly associated with buffer overflows,
is the most popular language with 17.1 percent of the market.
Embedded systems, network stacks, networked applications, and high-performance computing rely heavily upon C.


Address sanitizer is major buffer overflow prevention mechaniesm.
It works by xxx.


Because C/C++ focuses on performance and being close to the hardware,
there are gramma sugars that can do bounds checks on array references and pointer dereferences are not part of the language.

This can done with the help of compiler.
This should be calacualtes dynamicially.
If the user desires bounds checking, then for the memory accesses that cannot be resolved at compile time, the compiler must insert additional instructions into the generated code to perform the checks.


The time the processor takes to execute those extra instructions at run time is the performance overhead of a bounds checking mechanism.
The runtime overheade.



\yphn{Sanitizer Introduction}
Memory sanitizer refers to a wide range of techniques that ensure memory integrity and confidentiality,
i.e. memory access should not break memory layout unintentionally and read/write data regions unexpectedly.

There are a lot of memory sanitizers have been proposed.

Memory sanitizers are very useful.
ASan helps fuzzing tool found uncountable program bugs.

Memory sanitizers of course can detect memory errors that overwrite function pointers,
therefore they also can mitigate control-flow hijacking attacks.
Compared with control-flow integrity schemes,
memory sanitizers can prevent the attack one step earlier as they act before a malicious code point is activated.


\yphn{Problems of Address Sanitizer: Performance}
However, memory sanitizers are always complained about because of their higher performance overhead.

Asan introduces a performance overhead around two times.

I have found that many researchers report excellent low overheads for automated buffer overflow elimination schemes while in practical use they are actually substantially higher. For example, one set of researchers may report very low overheads by taking advantage of a trap on misaligned memory accesses, but many processors do not trap when dereferencing misaligned pointers. As a result, the check for misalignment must be performed by software, which causes a much higher overhead. Another set of researchers may use a shadow memory by dividing up the address space for the program and for the overflow checking. Again, this may not work in the practical domain because it requires operating system modifications.

We need practical solutons.
I investigated what performance could be achieved in the practical realm and how that performance could be improved for deployment of an automated, compiler-based memory safety checking tool.

\yphn{Existing Perf Optimization Schemes}
Use low-fat pointers. But introduce compatibility problems.
They should carefully support the mathematics operations on data pointers.

The overhead is mainly due to the high-frequency validity checking inside loops.

Move checks outside of loops. But there are no easy ways to identify complex loops.


\yphn{Our Scheme}
This work introduces a new scheme to reduce memory checking frequency,
it is orthogonal to existing schemes that move checkers outside of a loop.

The key idea is a hop-checking algorithm based on a red-zoned memory layout.
In this work, the concept of red-zone has the same meaning as it in the Asan,
i.e. data items are isolated with red-zones which are not allowed to read and write by vanilla program code.

Differently, in this work red-zones are filled with special data (suppose all 0xcc).
Therefore, invalid memory writes on red-zone areas would overwrite the special data,
thus can be detected by checking the integrity of red-zones.
Invalid reads would not leave any footprint, thus if want to detect them should check every memory read access on the spot.

Our scheme employs the principle of locality of data access.
Our hop-checking algorithm guarantees that memory accesses cannot cross the red-zones to destroy other data.
For each data pointer, we create a shadow to remember one of its historical values, i.e. whether it has already accessed.
A naive algorithm may not make use of the shadow value,
simply checking the data pointer once it is used to do memory access.
This algorithm works like the basic configuration of Asan, thus it would introduce high overhead.
On the other hand, our hop-checking algorithm checks the data pointer only when it moves forward or backward away from the shadow value beyond a threshold.
Suppose the threshold value is the size of a red-zone chunk, denoted as $REDZONE\_SIZE$.
Once the security checking is activated, it does security checks as most other address sanitizers do.
Meanwhile, it also backups the current value of the data pointer into the shadow,
preparing to trigger another checking in the next cycle.

Note that, we intentionally make the shadow not be initialized.
In such a way, it would store a randomized value, thus the data pointer always points to a memory address far away from such a  randomized value,
thus the data pointer would be checked at the first time it is used to access memory.

Briefly, every checking ensures that the following memory access would not cross over the red-zones to destroy other data regions.
However, they may read/write the red-zones.
As described above, writes to red-zones would be detected by checking the integrity of red-zones.
Reads to red-zones have a probability to escape from checking.
This can be mitigated by checking the last read access.

Because of the hop-skipping algorithm, we can reduce the frequency of memory checking, thus improve the performance.
This is very efficient for loops with small steps. For example, if the step is 1 as for a byte, and the $REDZONE\_SIZE$
equals to 32, then the overhead is reduced almost  32 times. Thus, it can improve the performance dramatically.

We use an LLVM pass to identify all the data pointers and create their shadows.


To shadow the variables,

We need to know the pointers.

It works like adjusting the source code. to enrich the source code.
1. How to identify data pointers;
2. How to instrument; base on LLVM; This should be accomplished by front-end.
3. How to do check;
4. The affection of compiler optimization.

In the front-end, we identify all the data pointers.
If the data pointer accesses memory, whether read or write, we create another pointer as a shadow.
This shadow is not used for access memory but used as a conditional trigger to do memory security checking.


Currently, We implementable this algorithm with source code.

\yphn{Why Asan High overhead}
For each memory access, it does checks?
Does it neccessary?

\yphn{How we reduce the overhead}
Access array elements and structu memmbers, the checking frequency can be reduced.

We improve th perforammcne by reduce checking freq3ucney to arrayes and structs.

Before teh checks, we
The core,

We depends a xxx on memory layout.
Data elements layout. Each elememt  with a trailing redzone.
For overflow.

How to do quick memory checking?

We create a red-zone around an element, the red-zone stores cookies (i.e. special data).
When there is overhead, it can be detected.
The data elelment are also aligned to a special address.
So we create element from different memroy region. just as low-fat.
So we can detect the start address with simple rounding down to a doubdanry.



Each checking ensure that current address is away from a red-zone.
So the


In this work with a memroy to the last checked memory address.
For each veariblae.

So for each memroy access, it can be deduced to an variable that access the emmroy.

For the value -set analsys. We know each instrauction would
access a set of memory address and a set of values.

Accodding to a compile top-down compiling process. SSA
This insruction is converted from a certain expression.
Which has a convertain semantic meansing in developoing langugen-level.
This memroy address?
access different data elementss at different iteration.

With historical knowledge. store the last checked adderss.
If the distance bwttten teh new addares and the lastst checked address is inside a bounary, i,..e. less than the size of
a red-zone.
We use a variable size of read-zone.
Or use a fixed size read-zone?
This is a trade of perforamnce with memory overhead. Currently, we use fixed size of redzones.
to reduce the complexity.

For each memory address access, we create a temoroy vairable to reambemer its last checked addresses.
When the progress on goning, if the new address is byond the gap, they we check it again.
around 2x memory ovrehead. The stride.

Why our method works? Because of array access?

It is easy for head varibles when we can control the memory allocator.
We can use a table to get its size,

If we use variable size of redzone, then how to create red-zone for stack buffers.

Use historcal values, we can avoid some problems.

\yphn{We need a profiling process}
We detect how does the memory access patterns.
How many memroy bufffers, how many objects.

Optimization for object access.

So we need some IR exampple to observe.


For loops, one is to hoist the memory check outside of loops.
But what about object access?

Can be hoist outside a loop.

Optimize object access checking.

An optimziation to address sanitizer:
Reduce the  checking fraeuncy.


We use variable size of redzones.Small memory overhead. 2x.
This can be achieved by hoist the checking outside of loops. But it difficult for xxx.
Each history address.

For object, we need that the object is checked.
$p->x$ and $p->y$ checked only once.

Why we can do it?
Why hoist checks outside loops? what is the challenged.

access data elements? How to hoist?

Do some kinds of tests before checking?

We also need to find out the challegnes to hoist checks outside loops?


Data memeber access?

For simplicity, we first ignore stack buffers.

There are my load and store instructions applied on the stack local variables.


% \begin{minted}[fontsize=\scriptsize,frame=single,linenos,xleftmargin=10pt,numbersep=-10pt]{cpp}
\begin{minted}[fontsize=\scriptsize,frame=single,xleftmargin=10pt]{cpp}
void foo(char *p) { while(*p){global_sum += *p; p++;} }
\end{minted}
The generated binary code contains two memory checks for the loop or optimiztion-level.
Altugho it can optimized to contail one check for optimize-level O2.
It is still.
In semantic, this is an array access. It can be optimized.




Expore the solutions to recude the frequence of checks.

Array access and object access.

Our naive optimization, can only optimize acces to arrarys.

But we provie optimizatin to objects.

Create Redzone inside an object. to detect internal overflow.

Maybe internal overflow is very importnt.
We can enable it.



If don't need to optimize.
Code usally contains If they are many access to diffiernt fields of the ame object in the same block.
So we can check it only.


We provide flags to control the optimzation.
If need to detect

How to deal with internal object overflow? ovverrite internal function pointers.
Lowfat cannot do it!


Existing DBI tool can count the excuted memory access, but they lack of the semantic information.
We want to know what it is accessed, access an array element, a pointer derefernece, or strucutre member.


\yphn{Contributions}
This work made the following contributions:

\begin{itemize}
  \item A new method to bounds-checking elimination: Symbolic Address-set analysis
  \item A symbolic execution engine on LLVM-IR.
  \item A prototype of semantic-aware address sanitizer.
  \item Detailed evaluations on SPECint.
\end{itemize}


To array access.
Try to hoist the check outside the loops.
However, it is not easy to identify.
The conditions are very harsh.

But for struct memory access, no optimiztion.

To do check at every memory access.

We can employ the checks with modifycation to
datalayout.

Memory access can be expected mode.


\yphn{Ground truth against Asan}

\yphn{Problem Analysis}

\yphn{Optimization suggestion}

\yphn{Problem of existing solutions}

\yphn{Our solution}


\yphn{xxx conclusion}

In addition to hoisting bounds checks out of loops, I also tried to hoist them out of functions. I reasoned that if I hoisted a bounds check into the caller, there might be information in the calling function to reveal the size of the objects, eliminating the need for a bounds check.


There is hope that automated buffer overflow checking will one day perform fast enough to work in future performance-critical systems.


Intel called "MPX" Memory Protection Extensions. Once Intel installs those hardware improvements, it is possible that we will be able to use compiler enforced buffer overflow elimination on performance-critical code as well.

The key problem is bound checking elimination.

\yphn{What kinds of elimination skills have alredy been applied on Asan}

The goal of this work is to reduce the checking frequency on accessing array elements and struct members.
The problem can be classified as boundary checking elimination. In fact, the goal is very simple:
```
if (isFarAway(p, p_lastchk)) {
    if (isOOB(p)) error();
    p_lastchk = p;
  }
```

Even though this problem sounds old, but not solved well.
Existing solutions that hoist the checking outside loops, or split the loops into several parts, do not work well.
These solutions try to be perfect, but once their assumptions do not meet, they do nothing.
Their flaws can be reflected by the implementation of Asan.
For the following code snippet. Asan checks each memory access.
```while(*p) {globalx +=*p, p++;}```

This project would apply IR-level symbolic address-set analysis, to collect symbolic-addresses accessed by each IR instruction.
Then use heuristic rules to find out patterns on accessing arrays and structs.
Once get all these patterns, we can eliminate checks accordingly.

The symbolic analysis algorithms are roughly designed according to my experience and early exploration.
Now, need to be implemented.


\yphn{Contributions}
This work made the following contributions:

\begin{itemize}
  \item A semantic-aware address sanitizing scheme which aims to reduce the bounds-checking frequency.  For array element access, struct member and pointer deference applies We care about three kinds of memory accesses and apply different policies on different access.
  \item A semantic-aware address sanitizer.
  \item A prototype implmentation of profiler.
  \item A prototype implementation of sanitizer.
  \item Detailed evaluations on SPEC.
\end{itemize}

We care about three kinds of memory accesses and apply different policies on different access.

\subsection{Asan \& Lowfat}
Therefore, we need to study Lowfat and Asan first.


% An IR instruction:
% {In-mems, out-mem, vout, vins, scope}

% Do we need to consider the scope?

% while() {
%     basic block;
%   }

% S_{inm}(I)
% S_{outm}(I)
% S_{mem}(I) = S_{inm}(I) U S_{outm}(I)

% V_{outv}(I)
% V_{inv}(I)
% V_{val}{I} = V_{outv}(I) U V_{inv}(I)

% S_{mem}{I} = {p, p+v, p+2*v, p+3*v}
% Arracy access:
% isFarAway(p, p_lastchk) && isOOB(ptr)

% S_{mem}{I} = {p, p+v, p+3v}
% S_{mem}{I'} = {p+v, p+2*v}


% Object access:
% isObj(ptr, type)

% Pointer defreence
% isOOB(ptr)


% init(func)
% for bb in func:
% bb = {}
% SI = {}



% for BB in F:
% if hasNewMS(BB):
%   execFunc(BB)

%   execFunc(BB)
%   {
%   for I in bb:
%   switch(op(I)) {
%       case op(I):
%       v_out = op(I, vins)
%       break;
%       case ...
%     }
%   V_{outv}.add{v_out}
%   v_{inm}.add{inm(I, v_ins)}
%   v_{outm}.add(outm(I, v_out))

%   S_{machine} = {R=xxx, M=xxx}

%       for suc in succssor(bb):
%       merge(SM_{suc}, S)
%     }



\begin{algorithm}
  \caption{Execute function $F$ with arguments $ARGS$}
  \footnotesize
  \begin{algorithmic}[1]
    \Function{ExecFunction}{$F$}
    \For{bb in F.BBs} \Comment{Initialize machine states for all basic blocks}
    \State bb.InStates = {$\emptyset$} \Comment{Current available input states}
    \State bb.OutSuperState = {$\emptyset$} \Comment{A super set of all output states}
    \EndFor
    \State F.entryBB.InStates.push(\{ARGS\}) \Comment{Create a state for entry block}
    \State

    \Do \Comment{Execute blocks which have new machine states}

    \State bCont = false
    \For{bb in F.BBs}
    \While {s = bb.InStates.pop()}
    \State ExecBasicBlock(bb, s)
    \State bCont = true;
    \EndWhile
    \EndFor

    \doWhile(bCont)
    \EndFunction
  \end{algorithmic}
\end{algorithm}


\begin{algorithm}
  \caption{Execute basic block $BB$ with machine state $S$}
  \footnotesize
  \begin{algorithmic}[1]
    \Function{ExecBasicBlock}{$BB$, $S$}
    \State outS = S.clone()  \Comment{Execute on a clone copy of $S$}
    \For{I in BB}

    \Switch{$I.opc$}
    \Case{$load$:}
    \State execInstrLoad(I, outS) \Comment{Execute $I$ and update $outS$}
    \EndCase
    \Case{$store$:}
    \State execInstrStore(I, outS)
    \EndCase
    \Case{$...$:}
    \State execInstr...(I, outS)
    \EndCase
    \EndSwitch

    \EndFor

    \State
    \State bNew = MergeState(BB, outS) \Comment{True if $outS$ contains new things}
    \If{bNew}
    \For{bb in BB.Succs} \Comment{Add $outS$ to successor blocks as input}
    \State bb.InStates.push(s.clone())
    \EndFor
    \EndIf

    \EndFunction
  \end{algorithmic}
\end{algorithm}


\begin{algorithm}
  \caption{Merge output state $S$ into $BB.OutSuperState$ }
  \footnotesize
  \begin{algorithmic}[1]
    \Function{MergeState}{$BB$, $S$}
    \State tmpSt = BB.OutSuperState.clone()
    \State tmpSt = tmpSt $\cup$ $S$ \Comment{Merge output state $S$}
    \State tmpSt = widenState(tmpSt) \Comment{Widening super output state}
    \If{tmpSt != BB.OutSuperState}
    \State BB.OutSuperState = tmpSt \Comment{state $S$ should carry new pattens}
    \State return true \Comment{Return true if $S$ carries new pattern knowledge}
    \EndIf
    \State return false \Comment{Otherwise, return false}
    \EndFunction
  \end{algorithmic}
\end{algorithm}



\section{Background and Related Works}

\newcounter{bkrw}
\renewcommand\yphn[1]{\stepcounter{bkrw} \textcolor{blue} {\thebkrw.} \yphl{#1}}

The following code snippet list the code generated by Asan.
It is in IR (intermediate representation) instructions instead of assembly code.
Although it does not exactly match the assembly code, the basic block level layout is almost the same.
Because IR contains rich semantic information.
Easier to understand the assembly code.

As the IR instructions show, they are a lot of memory access.
But most of them operate on local stack temporary variables, used as register spilling backup.
But most of them are not interesting.

\yphn{Array element access}


\begin{minted}[fontsize=\scriptsize,frame=single,linenos,numbersep=-6pt]{cpp}
  int access_array(int *p, int len) {
    for (int i = 0; i < len; i++) global_sum += p[i];
  }
\end{minted}


\begin{minted}[fontsize=\scriptsize,frame=single,linenos,numbersep=-6pt]{LLVM}
  ;sum += p[i];
  %2 = load i32*, i32** %p.addr, align 8
  %arrayidx = getelementptr i32, i32* %2, i64 %i
  %4 = load i32, i32* %arrayidx, align 4
  %5 = load i32, i32* %sum, align 4
  %add = add nsw i32 %5, %4
  store i32 %add, i32* %sum, align 4
\end{minted}

\yphn{Struct member access}


\begin{minted}[fontsize=\scriptsize,frame=single,linenos,numbersep=-6pt]{cpp}
  struct O { int x, y, z; };
  void access_object(struct O *obj) {
    global_sum += obj->x + obj->y + obj->z;
  }
\end{minted}


\begin{minted}[fontsize=\scriptsize,frame=single,linenos,numbersep=-6pt]{LLVM}
  ;sum += obj->x;
  ;sum += obj->y;
  %0 = load %struct.O*, %struct.O** %obj.addr, align 8
  %x = getelementptr %struct.O, %struct.O* %0, i32 0, i32 0
  %1 = load i32, i32* %x, align 4
  %2 = load i32, i32* %sum, align 4
  %add = add nsw i32 %2, %1
  store i32 %add, i32* %sum, align 4
  %3 = load %struct.O*, %struct.O** %obj.addr, align 8
  %y = getelementptr %struct.O, %struct.O* %3, i32 0, i32 1
  %4 = load i32, i32* %y, align 4
  %5 = load i32, i32* %sum, align 4
  %add1 = add nsw i32 %5, %4
  store i32 %add1, i32* %sum, align 4
\end{minted}


\yphn{Related works}

I began with two memory safety checkers that meet these criteria, SAFECode and SoftBound.

\subsection{Peroformance Improvment}

I made three performance enhancements to SoftBound to investigate their effects on performance.
~\cite{Performance of Compiler-Assisted Memory Safety Checking}

First, I hoisted spatial memory access checks out of loops when the loop bounds were known on entry. As an example, consider the following function:


I next hoisted bounds checks out of a function and into its callers when the compiler could see all calls to the function, so that a bounds check will be executed somewhere (if necessary) if it is deleted from its original function.

I found that unrolling loops thwarted some attempts to hoist bounds checks.

Fully unrolled loops contained a sequence of memory accesses in straight-line code in place of one access within a loop. I therefore disabled loop unrolling.

The third change was to test the performance of bounds checks on stores only (to prevent arbitrary code execution), or on strings only (because incorrect string management is a leading cause of vulnerabilities), or only on stores to strings. Limiting the bounds checks in this way can provide some insight into the tradeoff between security and performance.


\subsection{Asan}
\yphn{What is the idea}
1. The run-time library replaces the \textit{malloc()} and \textit{free()} functions.
2. The memory around malloc-ed regions (red zones) is poisoned.
3. The free-ed memory is placed in quarantine and also poisoned.
4. Every memory access in the program is checked if access a posied address.

So what the different between positon and red-zone?

This is guaranteed by the fact that malloc returns 8-byte aligned chunks of memory.

malloc allocates the requested amount of memory with redzones around it. The shadow values corresponding to the redzones are poisoned and the shadow values for the main memory region are cleared.

free poisons shadow values for the entire region and puts the chunk of memory into a quarantine queue (such that this chunk will not be returned again by malloc during some period of time).


\yphn{The instruemntation and datalayout}
It consisst of a compiler insttrumentation pass and
a runtime library tha rep;ace the malloc funcitons.

The tricky part is how to implement IsPoisoned very fast and ReportError very compact.

If the shadow bytes is not zeor, which meanshing some bytesthe corspoinding program memory adddress
are poisend. Then it will check wit a slow path.

There is a shadowgap, which is not accessable.
Any access woul caseu the progarm crash.

The mappinf fucntiohn is very simiple
$Shadow = (Mem >> 3) + 0x7fff8000;$

How about use morey use even more compact shadow memory.
Every access to porgram memory would companied with a shoawod memory access.
This is the root cause of overhead.

For unaligned access.The current compact mapping will not catch unaligned partially out-of-bound accesses:


\yphn{Semantic-aware or not?}
\yphn{Input and Output}
\yphn{Optimizations}

Call stack & incompatibility
To ignore certain functions, one can use the no_sanitize_address attribute
supported by Clang and GCC.
Or crate and use a blacklist.

\yphn{Momorey overhead}
because ASan consumes 20 terabytes of virtual memory (plus a bit).


\yphn{Optimization}
AddressSanitizer does not need to instrument all memory accesses to find all bugs.

1. avoid instrumenting individual accesses. hoist the checks outside of loops.
2. Combine two accesses into one for struct member access.

4. Using dataflow techniques for range checks elimination

Combining checks across loop iterations:

\begin{minted}[fontsize=\scriptsize,frame=single,linenos,numbersep=-6pt]{cpp}
  int *p;
  // Enough to check __asan_region_is_poisoned(a, n*sizeof(*a))
  for (i = 0; i < n; ++i)
    p[i];
\end{minted}

The virtual address space is diviedde into 2 disjoiont classes:
program memory and shadow memroy.
Progarm memory is used by the regular program code.
While shadows memory contains the metadata and shaow values.

These two classess so mroy are well orgarnize such that the mapping progarm addrss
to shadow address con be done fast.

ASan maps 8 bytes of the progarm memory into 1 buytes of the shadow memory.

Out-of-bounds accesses to heap, stack and globals
Double-free, invalid free


\yphn{Overhead}
~2x

\yphn{Security Guarantees}
It can detect headp and stack buffer ovrealow.
Also can do other things including user after frer and sue after return.



\subsection{LowFat}


\subsection{   Range check elimination}
// The InductiveRangeCheckElimination pass splits a loop's iteration space into
// three disjoint ranges.  It does that in a way such that the loop running in
// the middle loop provably does not need range checks. As an example, it will
// convert


\subsection{Bounds-checking elimination}
In computer science, bounds-checking elimination is a compiler optimization useful in programming languages or runtime systems that enforce bounds checking, the practice of checking every index into an array to verify that the index is within the defined valid range of indexes.[1] Its goal is to detect which of these indexing operations do not need to be validated at runtime, and eliminating those checks.

So behind this work is a boundary chekcing eliminattion.

\yphn{Various boundary checks elimination algorithms}




\section{System Design}


\newcounter{dsgn}
\renewcommand\yphn[1]{\stepcounter{dsgn} \textcolor{blue} {\thedsgn.} \yphl{#1}}


We design a semantic-aware memory access profiler and
a semantic-aware address sanitizer.

\subsection{Languate Level Semantics}
To parse the semantics of each memory access is critical.

Currently, we define the following semantics: array access or struct access, pointer dereference.

\yphn{Inputs}

We exploit the source code.

\yphn{Outputs}
For a program, both profiler and sanitizer would generate their binaries.


\subsection{Semantic Analysis}
An important task is to collect the semantics of memory access.

\subsection{Memory object organization}
Do we need to use variable size red-zones for different kinds of objects?

\subsubsection{Heap Buffers}

\subsubsection{Stack buffers}

\subsubsection{Internal buffers}
An object sometimes contains a buffer inside itself.
This buffer may be security-sensitive if its adjacent members are function pointers.

For such an object, we create red-zones to surround the buffers, similar to what Asan does on stack buffers.

\subsection{Instrumentation}

Do we need to do another check after a loop?

\subsection{Exception Handling}


\begin{minted}[fontsize=\scriptsize,frame=single,linenos,numbersep=-6pt]{llvm}
  entry:
    %p.addr = alloca i8*, align 8
    store i8* %p, i8** %p.addr, align 8
    br label %while.cond  
  while.cond:
    %0 = load i8*, i8** %p.addr, align 8
    %1 = load i8, i8* %0, align 1
    %tobool = icmp ne i8 %1, 0
    br i1 %tobool, label %while.body, label %while.end
  while.body:    
    %2 = load i8*, i8** %p.addr, align 8
    %3 = load i8, i8* %2, align 1
    %conv = sext i8 %3 to i32
    %4 = load i32, i32* @globalx, align 4
    %add = add nsw i32 %4, %conv
    store i32 %add, i32* @globalx, align 4
    %5 = load i8*, i8** %p.addr, align 8
    %incdec.ptr = getelementptr inbounds i8, i8* %5, i32 1
    store i8* %incdec.ptr, i8** %p.addr, align 8
    br label %while.cond  
  while.end:
    ret void
\end{minted}

A sound algorithm.

Symbolic address-set Analsysis based on Symbolic IR execution.

This is really an old problem.



\subsection{Symbolic IR Execution}

% The design of symbolic execution engine
% Input & output

The address set of each IR register.
LLVM-IR has unlimited registesrs.

Algorithm:
The exeuction enginne schedule the execution basic block by block.
The schedule units are basiuc block.
Each basic block are feed by a seriours of configuartion.
Each configuration is consist with a set of rgistesrs status.
The entry basick block start with an empty basic block.
The partermeters.
Eveay new occurentcy of a regiuster is denoted by a symbol value.
Every IR instructoin is.
The register staus changes along with the IR execution.

Machine status: Register status. + Memory status.

Change the status along the IR execution.

All registesrs and memroy slots are initialized with symbolic values.
Therefore, the first read of a registe or a memoery adddress would get a symbolic values.

Every Function is an execution unit.




\subsection{Symbolic Address-set Analysis}
For IR instruction, simply idenfitied by the line number.
On my symbolic execution engine, collect the accessed address by each instruction.

Then analsys the memory access patterns, to
idenitfy array access an struct member access.




\subsection{Instrumentation}

\subsection{Data Layout}






% An example of a floating figure using the graphicx package.
% Note that \label must occur AFTER (or within) \caption.
% For figures, \caption should occur after the \includegraphics.
% Note that IEEEtran v1.7 and later has special internal code that
% is designed to preserve the operation of \label within \caption
% even when the captionsoff option is in effect. However, because
% of issues like this, it may be the safest practice to put all your
% \label just after \caption rather than within \caption{}.
%
% Reminder: the "draftcls" or "draftclsnofoot", not "draft", class
% option should be used if it is desired that the figures are to be
% displayed while in draft mode.
%
%\begin{figure}[!t]
%\centering
%\includegraphics[width=2.5in]{myfigure}
% where an .eps filename suffix will be assumed under latex, 
% and a .pdf suffix will be assumed for pdflatex; or what has been declared
% via \DeclareGraphicsExtensions.
%\caption{Simulation Results}
%\label{fig_sim}
%\end{figure}

% Note that IEEE typically puts floats only at the top, even when this
% results in a large percentage of a column being occupied by floats.


% An example of a double column floating figure using two subfigures.
% (The subfig.sty package must be loaded for this to work.)
% The subfigure \label commands are set within each subfloat command, the
% \label for the overall figure must come after \caption.
% \hfil must be used as a separator to get equal spacing.
% The subfigure.sty package works much the same way, except \subfigure is
% used instead of \subfloat.
%
%\begin{figure*}[!t]
%\centerline{\subfloat[Case I]\includegraphics[width=2.5in]{subfigcase1}%
%\label{fig_first_case}}
%\hfil
%\subfloat[Case II]{\includegraphics[width=2.5in]{subfigcase2}%
%\label{fig_second_case}}}
%\caption{Simulation results}
%\label{fig_sim}
%\end{figure*}
%
% Note that often IEEE papers with subfigures do not employ subfigure
% captions (using the optional argument to \subfloat), but instead will
% reference/describe all of them (a), (b), etc., within the main caption.


% An example of a floating table. Note that, for IEEE style tables, the 
% \caption command should come BEFORE the table. Table text will default to
% \footnotesize as IEEE normally uses this smaller font for tables.
% The \label must come after \caption as always.
%
%\begin{table}[!t]
%% increase table row spacing, adjust to taste
%\renewcommand{\arraystretch}{1.3}
% if using array.sty, it might be a good idea to tweak the value of
% \extrarowheight as needed to properly center the text within the cells
%\caption{An Example of a Table}
%\label{table_example}
%\centering
%% Some packages, such as MDW tools, offer better commands for making tables
%% than the plain LaTeX2e tabular which is used here.
%\begin{tabular}{|c||c|}
%\hline
%One & Two\\
%\hline
%Three & Four\\
%\hline
%\end{tabular}
%\end{table}


% Note that IEEE does not put floats in the very first column - or typically
% anywhere on the first page for that matter. Also, in-text middle ("here")
% positioning is not used. Most IEEE journals/conferences use top floats
% exclusively. Note that, LaTeX2e, unlike IEEE journals/conferences, places
% footnotes above bottom floats. This can be corrected via the \fnbelowfloat
% command of the stfloats package.

\section{Evaluation}


\begin{table*}[]
    \begin{tabular}{|c|c|c|c|c|c|c|c|c|c|c|}
        \hline
              & \multicolumn{5}{c|}{Arrays (data arrays and object arrays)} & \multicolumn{5}{c|}{Struct instances \& Class objects}                                                                                                                    \\ \hline
              & Decl-locs                                                   & Min-eles                                               & Max-eles & \#Static-access & \#Dyn-access & Decl-locs & Min-fileds & Max-fields & \#Static-access & \#Dyn-access \\ \hline
        Prog1 &                                                             &                                                        &          &                 &              &           &            &            &                 &              \\ \hline
        Prog2 &                                                             &                                                        &          &                 &              &           &            &            &                 &              \\ \hline
        Prog3 &                                                             &                                                        &          &                 &              &           &            &            &                 &              \\ \hline
    \end{tabular}
\end{table*}


\section{Conclusion}





% conference papers do not normally have an appendix


% use section* for acknowledgement
% \section*{Acknowledgment}

% The authors would like to thank...





% trigger a \newpage just before the given reference
% number - used to balance the columns on the last page
% adjust value as needed - may need to be readjusted if
% the document is modified later
%\IEEEtriggeratref{8}
% The "triggered" command can be changed if desired:
%\IEEEtriggercmd{\enlargethispage{-5in}}

% references section

% can use a bibliography generated by BibTeX as a .bbl file
% BibTeX documentation can be easily obtained at:
% http://www.ctan.org/tex-archive/biblio/bibtex/contrib/doc/
% The IEEEtran BibTeX style support page is at:
% http://www.michaelshell.org/tex/ieeetran/bibtex/
\bibliographystyle{IEEEtranS}
% argument is your BibTeX string definitions and bibliography database(s)
\bibliography{IEEEabrv,refs}
%
% <OR> manually copy in the resultant .bbl file
% set second argument of \begin to the number of references
% (used to reserve space for the reference number labels box)
% \begin{thebibliography}{1}

% \bibitem{IEEEhowto:kopka}
% H.~Kopka and P.~W. Daly, \emph{A Guide to \LaTeX}, 3rd~ed.\hskip 1em plus
%   0.5em minus 0.4em\relax Harlow, England: Addison-Wesley, 1999.

% \end{thebibliography}




% that's all folks
\end{document}


